%-------------------------
% Cover Letter in LaTeX
% Formatting based on Resume
%------------------------

\documentclass[letterpaper,11pt]{article}

\usepackage{latexsym}
\usepackage[empty]{fullpage}
\usepackage{titlesec}
\usepackage{marvosym}
\usepackage[usenames,dvipsnames]{color}
\usepackage{verbatim}
\usepackage{enumitem}
\usepackage[colorlinks=true, linkcolor=linkblue, urlcolor=linkblue, citecolor=linkblue]{hyperref}
\usepackage{fancyhdr}
\usepackage[english]{babel}
\usepackage{tabularx}
\usepackage{fontawesome5}
\usepackage[charter]{mathdesign}
\usepackage{xcolor}
\input{glyphtounicode}

% Define custom blue colors to match PDF
\definecolor{headerblue}{RGB}{37,150,190}
\definecolor{linkblue}{RGB}{0,0,238}

% EDIT THESE VARIABLES
\newcommand{\companyname}{[Company Name]}
\newcommand{\positionname}{[Position Name]}

\pagestyle{fancy}
\fancyhf{}
\fancyfoot{}
\renewcommand{\headrulewidth}{0pt}
\renewcommand{\footrulewidth}{0pt}

% Adjust margins
\addtolength{\oddsidemargin}{-0.5in}
\addtolength{\evensidemargin}{-0.5in}
\addtolength{\textwidth}{1in}
\addtolength{\topmargin}{-.5in}
\addtolength{\textheight}{1.0in}

\urlstyle{same}

\raggedbottom
\raggedright
\setlength{\tabcolsep}{0in}

% Sections formatting
\titleformat{\section}{
  \vspace{-3pt}\scshape\raggedright\large\color{black}
}{}{0em}{}[\color{headerblue}\titlerule \vspace{-5pt}]

\pdfgentounicode=1

\begin{document}

%----------HEADING----------
\begin{center}
    \textbf{\Huge \scshape Achal Patel} \\ \vspace{2pt}
    \small \faPhone\ 438-979-5673 $|$ 
    \href{mailto:Achalypatel3403@gmail.com}{\faEnvelope\ \underline{Achalypatel3403@gmail.com}} $|$ 
    \href{https://linkedin.com/in/achal-patel}{\faLinkedin\ \underline{linkedin.com/achal-patel}} $|$
    \href{https://github.com/ac-pate}{\faGithub\ \underline{github.com/ac-pate}}
\end{center}

\vspace{0.4in}

\today

\vspace{0.15in}

Hiring Manager \\
\textbf{\companyname}

\vspace{0.2in}

\textbf{Re: Application for \positionname}

\vspace{0.15in}

To Whom It May Concern,
\vspace{0.2in}

I am currently co-leading the MIMIC Capstone project with McGill's Mobile Robotics Lab, where we are training Vision-Language-Action (VLA) models and behavior cloning policies like ACT and diffusion for bimanual mobile manipulation. We are collecting real-world demonstration data, training custom models from scratch, and deploying them on physical hardware to enable robots to learn complex manipulation tasks from human demonstrations. Beyond this, I have designed a custom Transformer-CNN hybrid architecture that won a Kaggle competition for RNA folding prediction, beating PhD and Master's students. I have also completed projects in sensor fusion (Extended Kalman Filters for vehicle tracking) and am currently studying machine learning for health applications. My work on autonomous sumobots involved deploying OpenCV and YOLO in real-time for opponent detection during robotic combat. I am applying to \companyname\ because your approach to machine learning aligns with my focus: building models that operate in the real world, not just in notebooks.
\vspace{0.2in}

I bring hands-on experience with the full ML pipeline. I train models using PyTorch and deploy them on edge devices (Jetson AGX Xavier, Jetson Orin Nano). I work extensively with Hugging Face tools for model deployment, manage data collection pipelines for imitation learning, and have implemented state estimation algorithms for robotics. I am a CEC National Champion, former VP of Projects at IEEE Concordia (leading projects across IoT, drones, and ML-based systems), and I have delivered technical workshops to students. My portfolio spans computer vision, reinforcement learning, sensor fusion, and generative models. I execute quickly, iterate relentlessly, and build systems that perform under real-world constraints. I would bring immediate value to your team.

\vspace{0.15in}

Sincerely,

\vspace{0.1in}

Achal Patel

\end{document}
